\documentclass[a4paper]{article}

\usepackage[english]{babel}
\usepackage[utf8]{inputenc}
\usepackage{amsmath}
\usepackage{graphicx}
\usepackage[colorinlistoftodos]{todonotes}

\title{Special Relativity}

\author{You}

\date{\today}

\begin{document}
\maketitle
%Section: Website Layout
\section{Website Layout}
Each subsection here to be a different tab up top. Note: Can Combine Lorentz Transformations and Consequences
\subsection{Introduction}
Inludes section on Introduction + Gallilean Transformations
\subsection{Einstein}
Contains Einstein section + some other history
\subsection{Lorentz Transformations}
Includes section on Lorentz Transformations
\subsection{Consequences}
Includes Time Dilation and Length Contraction
\subsection{Quiz}
Includes Quiz
\subsection{About}
Includes author bios and citations

%Section: Introduction
\section{Introduction}
What is Special Relativity? Special Relativity is a theory which connects space and time. It has been experimentally verified (Insert citation) and is the current theory connecting the two.\\

To begin understanding Special Relativity, we must first declare two very important definitions. These definitions will be crucial to our understanding of Special Relativity.
\subsection{Event}
An event is something that happens independent of the reference frame we might use to describe it.
\subsection{Observer}
An observer is an infinite set of clocks, on at each point in space, at rest and synchronized with respect to eachother.\\
The space-time coordinates (x,y,z,t) are recorded by a clock at the location (x,y,z) of the event at the time (t) it occurs.

%Section: Gallilean Tranfsformations
\section{Gallilean Transformations}
We have two reference frames, S and S$^\prime$ with S$^\prime$ moving relative to S with constant velocity $v$.\\

TODO:INSERT PICTURE OF REFERENCE FRAMES  %http://en.wikipedia.org/wiki/File:Frames_of_reference_in_relative_motion.svg)

\begin{center}

$x^\prime = x-vt$ \\
$y^\prime = y$ \\
$z^\prime = z$\\
$t^\prime = t$\\

\end{center}
Why do Gallilean Transformations not work? Lets consider an example:

\begin{center}
TODO: INSERT IMAGE OF TRAIN
\end{center}

From this, we can see that the light reaching the front and the light reaching the back are not simulatanious.

%Section: Einstein
\section{Einstein}
In Albert Einstein's famous paper "On the Electrodynamics of Moving Bodies", written in 1905 he proposed two postulates:\\
1. That the laws of physics are invariant (identical) in all inertial systems (non-accelerating frames).\\
2. The speed of light in a vacuum is the same for all observers, regardless of motions.\\




%Section: Lorentz Transformations
\section{Lorentz Transformations}
Similar to our Gallielean Transformation, let us consider two reference frames, S and S$^\prime$ with S$^\prime$ moving relative to S with constant velocity $v$. \\
Let $t=0$ and $t^\prime =0$ to be when the origins of S and $S^\prime$ coincide.
Using the fact that the transformation between S and $S^\prime$ must be linear we see:
\begin{center}

$ x^\prime = a_{11}x + a_{12}y + a_{13}z + a_{14}t$\\
$ y^\prime = a_{21}x + a_{22}y + a_{23}z + a_{24}t$\\
$ z^\prime = a_{31}x + a_{32}y + a_{33}z + a_{34}t$\\
$ t^\prime = a_{41}x + a_{42}y + a_{43}z + a_{44}t$\\
\end{center}
Why must they be linear? Let's consider:

Suppose $x^\prime = ax^2$\\
Then $x_2^\prime - x_1^\prime = ax_2^2 - ax_1^2$\\
$x_2^\prime - x_1^\prime = a(x_2^2 - x_1^2)$\\
Consider $x_2 = 2$ and $x_1 = 1$\\
$x_2^\prime - x_1^\prime = a(4 - 1)$\\
$x_2^\prime - x_1^\prime = a(3)$\\
Now consider $x_2 = 5$ and $x_1 = 4$\\
$x_2^\prime - x_1^\prime = a(25 - 16)$\\
$x_2^\prime - x_1^\prime = a(9)$\\

From this, we can see that depending on where we choose our origin, we get a different result. Clearly this is incorrect.

Let's look again at the postulate of the constancy of the speed of light. Imagine that at a time $t=0$, a spherical light beam leaves a point that coincides to both the origin in $S$ and the origin in $S^\prime$.\\
We know that the light will travel following the following formulas:
\begin{center}
$x^2 + y^2 + z^2 = (ct)^2$ in our S frame\\
$(x^\prime)^2 + (y^\prime)^2 + (z^\prime)^2 = (ct^\prime)^2$ in our $S^\prime$ frame.
\end{center}
If we take our equations for a linear transformation and plug them into these two equations we receive the following:
\begin{center}
$x^\prime = \frac{x - vt}{\sqrt{1-\frac{v}{c}^2}}$\\
$y^\prime = y$\\
$z^\prime = z$\\
$t^\prime = \frac{t - \frac{v}{c^2}x}{\sqrt{1-\frac{v}{c}^2}}$\\
and\\
$x= \frac{x^\prime - vt^\prime}{\sqrt{1-\frac{v}{c}^2}}$\\
$y= y$\\
$z= z$\\
$t= \frac{t^\prime - \frac{v}{c^2}x}{\sqrt{1-\frac{v}{c}^2}}$\\
\end{center}
These here are the famous Lorentz Transformations.\\
You'll notice that with $v<<c$ that $\sqrt{1-\frac{v}{c}^2} \simeq 1$ as $\frac{v}{c} -> 0$ This reduces our Lorentz Transformations to our classic Gallilean Transformation for speeds much less than the speed of light. Which would be everything we encounter on a day to day basis.

%Section: Length Contraction
\section{Length Contraction}

Let's consider a rod at rest in the S frame.\\
S measures endpoints at $x_1$ and $x_2$\\
$S^\prime$ measures end points at $x^\prime_1$ and $x^\prime_2$\\
To measure these lengths, we must ensure the measurements were performed at the same time, a time $t=t_0$\\
\begin{center}
$x^\prime_2 - x^\prime_1 = \frac{x_2 - vt_0}{\sqrt{1-(\frac{v}{c})^2}} - \frac{x_1 - vt_0}{\sqrt{1-(\frac{v}{c})^2}}$\\
$x^\prime_2 - x^\prime_1 = \frac{x_2 - x_1}{\sqrt{1-(\frac{v}{c})^2}}$\\
$(x_2 - x_1) = (x^\prime_2 - x^\prime_1) (\sqrt{1-(\frac{v}{c})^2})$\\
$\Delta x = (\Delta x^\prime) (\sqrt{1-(\frac{v}{c})^2})$
\end{center}

When a body is moving at velocity $v$ relative to a stationary observer its length measured is contracted in the direction of the motion by a factor of $(\sqrt{1-(\frac{v}{c})^2})$, that is a body's length is measured to be greatest when it is at rest relative to the observer. \\
Note on Length contraction, the length is only contracted in the direction of the relative motion.

%Section: Time Dilation
\section{Time Dilation}
We have a clock at rest at a position $x^\prime_0$ which is in the $S^\prime$ frame. In our $S^\prime$ frame measures times $t^\prime_2$ and $t^\prime_1$ and in our S frame measures times $t_2$ and $t_1$.\\
\begin{center}
$t_2 - t_1 = \frac{t^\prime_2 - \frac{v}{c^2}x_0}{\sqrt{1-\frac{v}{c}^2}} - \frac{t^\prime_1 - \frac{v}{c^2}x_0}{\sqrt{1-\frac{v}{c}^2}}$\\
$t_2 - t_1 = \frac{t^\prime_2 - t^\prime_1}{\sqrt{1-\frac{v}{c}^2}}$\\
$\Delta t = \frac{\Delta t^\prime}{\sqrt{1-\frac{v}{c}^2}}$\\
\end{center}
In other words, a clock goes fastest when it is at rest. When the clock moves with velocity $v$ it appears to have slowed down by a factor $(\sqrt{1-(\frac{v}{c})^2})$.

%Section: Quiz
\section{Quiz}
In this section we could have 3-4 questions relating to Special Relativity for the students to test their knowledge. It would be a separate page with (hopefully) interactive feedback.

%Section: About
\section{About}
Eric Lessard is - INSERT\\
Dragan Lukic is -INSERT\\
Anmol Dhar is -INSERT\\
Aaron Bondy is -INSERT\\

%Section: References
\section{References}
1. E.H Kim, Lecture Notes on Mechanics (64-250) Recorded 2014\\
2. Einstein Albert (1905) "Zur Elektrodynamik bewegter Korper". (English Translation by George Barker Jeffery and Wilfrid Perrett (1923).\\
3, Reference frame picture from Wikipedia user Krea licensed under Creative Common License\\
% http://en.wikipedia.org/wiki/File:Frames_of_reference_in_relative_motion.svg
4. Train example picture from some .edu will find later\\
5.

\end{document}